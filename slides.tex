\documentclass[10pt]{beamer}
\usetheme{Frankfurt}
\usepackage{graphicx}
\usepackage{amsmath}
\usepackage{hyperref}

% Title page settings
\title{Introduction to Remote Sensing}
\subtitle{Applications in Disaster Management}
\author{Manuel Arturo Suárez Améndola}
\date{\today}

\begin{document}

% Title slide
\begin{frame}
    \titlepage
\end{frame}

% Table of Contents
\begin{frame}{Outline}
    \tableofcontents
\end{frame}

% Section 1: Introduction
\section{Introduction to Remote Sensing}
\begin{frame}{What is remote sensing}
  \begin{block}{Remote Sensing}
    Remote sensing is the process of detecting and monitoring the physical characteristics of an area
    by measuring its reflected and emitted radiation at a distance.
  \end{block}
    \begin{itemize}
        \item Cameras on satellites and airplanes take images of large areas on the Earth's surface.
        \item Sonar systems on ships can be used to create images of the ocrean floor without needing to travel to the bottom of the ocean.
        \item Cameras on satellites can be used to make images of temperature changes in the oceans.
    \end{itemize}
\end{frame}

\begin{frame}{What is it used for?}
  Some specific users of remotely sensed images of the Earth includes:

  \begin{itemize}
    \item Large forest fires can be mapped from space, allowing rangers to ses a much larger area than from the ground
    \item Tracking clouds to help predict the weather or watching erupting volcanoes, and help watching for dust storms
    \item Tracking the growth of a city and changes in farmland or forests over several years or decades
    \item Discovery and mapping of the rugged topograhpy of the ocean floor
  \end{itemize}
\end{frame}

% Section 2: Machine Learning Techniques
\section{Machine Learning Techniques in Disaster Management}
\begin{frame}{Machine Learning Techniques}
    \begin{itemize}
        \item \textbf{Support Vector Machines (SVMs)}: Used for classifying flooded regions or landslide areas.
        \item \textbf{Random Forest (RF)}: Detects wildfire risk zones by analyzing land cover changes.
        \item \textbf{K-Nearest Neighbors (KNN)}: Identifies disaster-affected areas by comparing with historical data.
    \end{itemize}
    \vspace{5mm}
    \textbf{Applications}: Flood detection, wildfire prediction, earthquake damage assessment.
\end{frame}

% Section 3: Deep Learning Techniques
\section{Deep Learning Techniques in Disaster Management}
\begin{frame}{Deep Learning Techniques}
    \begin{itemize}
        \item \textbf{Convolutional Neural Networks (CNNs)}: Used for image segmentation to detect floods, landslides, or damaged buildings.
        \item \textbf{Recurrent Neural Networks (RNNs)}: Time-series analysis for wildfire spread or flood progression.
        \item \textbf{Generative Adversarial Networks (GANs)}: Enhancing satellite imagery quality, disaster simulation.
    \end{itemize}
\end{frame}

% Section 4: SAR Texture and Polarimetric Analysis
\section{SAR Characteristics and Analysis}
\subsection{SAR Texture Analysis}
\begin{frame}{Texture in SAR Images}
    \begin{itemize}
        \item Texture refers to spatial patterns in pixel intensity.
        \item Common texture features: Contrast, Entropy, Homogeneity, Energy.
        \item \textbf{Applications}: 
        \begin{itemize}
            \item Flood detection (smooth texture in water regions).
            \item Landslide and urban damage detection.
        \end{itemize}
    \end{itemize}
\end{frame}

\subsection{Polarimetric SAR Analysis}
\begin{frame}{Polarimetric SAR (PolSAR) Analysis}
    \begin{itemize}
        \item Polarimetric SAR captures data in multiple polarization states (HH, HV, VV, VH).
        \item **Scattering mechanisms**: Surface, double-bounce, and volume scattering.
        \item \textbf{Decomposition Techniques}:
        \begin{itemize}
            \item \textbf{Cloude-Pottier Decomposition}
            \item \textbf{Freeman-Durden Decomposition}
            \item \textbf{H/A/ Decomposition}
        \end{itemize}
        \item \textbf{Applications}:
        \begin{itemize}
            \item Flood mapping: PolSAR identifies flooded areas by distinguishing surface scattering.
            \item Oil spill detection: Differentiates oil-covered water from clean water.
            \item Earthquake damage: Monitors building collapse via double-bounce scattering.
        \end{itemize}
    \end{itemize}
\end{frame}

% Section 5: Case Studies
\section{Case Studies}
\begin{frame}{Case Studies of ML/DL in Disaster Response}
    \begin{itemize}
        \item \textbf{Hurricane Damage Assessment}: CNNs used to detect flood extent post-Hurricane Harvey.
        \item \textbf{Nepal Earthquake}: SAR and CNNs used to detect landslides.
        \item \textbf{California Wildfires}: RF models for predicting fire-prone areas and DL for burn severity mapping.
    \end{itemize}
\end{frame}

% Section 6: Conclusion
\section{Conclusion}
\begin{frame}{Conclusion}
    \begin{itemize}
        \item Machine learning and deep learning are transforming disaster management through fast and accurate satellite image analysis.
        \item SAR texture and polarimetric analysis provide detailed surface information for disasters like floods, oil spills, and earthquakes.
        \item The integration of AI and remote sensing will enhance real-time disaster detection and response capabilities.
    \end{itemize}
\end{frame}

% References slide
\section{References}
\begin{frame}{References}
    \begin{itemize}
        \item \url{https://disasterscharter.org/}
        \item Sentinel-1 and Sentinel-2 data examples from \url{https://www.copernicus.eu/}
        \item PRISMA hyperspectral satellite data from \url{https://www.asi.it/en/activity/earth-observation/prisma/}
    \end{itemize}
\end{frame}

\end{document}

